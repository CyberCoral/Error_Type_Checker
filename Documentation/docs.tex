\documentclass{article}
\usepackage[T1]{fontenc}
\usepackage{listings}
\usepackage{hyperref}
\hypersetup{
	colorlinks=true,
	linkcolor=blue,
	filecolor=magenta,      
	urlcolor=cyan,
	pdftitle={Overleaf Example},
	pdfpagemode=FullScreen,
}

\usepackage{xcolor}

\definecolor{grey}{rgb}{0.1,0.3,0.2}
\definecolor{purpell}{rgb}{0.3,0.5,0.7}
\definecolor{string_green}{rgb}{0,206,0}

\DeclareFixedFont{\ttb}{T1}{txtt}{bx}{n}{9} % for bold
\DeclareFixedFont{\ttm}{T1}{txtt}{m}{n}{9}  % for normal


\begin{document}
	\begin{titlepage}
		\centering
		\vfill
		\vfill
		{\huge Documentation of Automated\_Error\_Checks.py}
		\vspace{3cm}
		
		by
		{\Large 
		CyberCoral
		\vspace{2cm}
		
		\today}
		\vfill

	\end{titlepage}
	\newpage
	\tableofcontents
	\newpage
	\lstset{language=python,
		tabsize=2,
		commentstyle=\color{grey},    % comment style
		stringstyle=\color{purpell},  % string literal style
		keywordstyle=\color{blue},
		emphstyle=\color{red}    % Custom highlighting style
	}
	\section{Introduction to the program.}
	The program, written in Python3 and using the regex module \textbf{re}, is a python error handler for general use. It means this program will let you cover more easily each condition and type check you would need.
	
	Instead of using various if-else statements, which occupy many lines of code, you can use that module and its functions to compact that part of any secure code.
	
	\lstinputlisting[language=Python, firstline=0, lastline=9,caption={Credits for the author.}]{Automated_Error_Checks.py}
	
	There are three main sections that this documentation will cover:
	
	\subsection{Simple functions.} 
	In this section, the basics of the program will be covered. It will view two main programs: 
	
	ErrorTypeFinder() and ConditionCheck(). \ref{error_check_1}
	
	\hspace{-0.5cm}It also will show a quality of life converter called BinaryToTruthList(). \ref{binary}
	
	\ref{simple}
	
	\subsection{Complex / Automated functions.}
	In this part, there are the advanced versions of the three basic functions, expanded to cover more than 1 variable at a time. 
		
	\ref{complex}
	\subsection{Use in Python3 example.}
	Last but not least, there will be an example of how you can use the module and its functions to save code.
	
	\ref{example}
	\newpage
	\section{Simple functions: BinaryToTruthList(), ErrorTypeFinder(), ConditionCheck()} 
	\label{simple}
	
	The core functions of the module are ErrorTypeFinder() and ConditionCheck():
	
	\subsection{ErrorTypeFinder()}
		
	\lstinputlisting[language=Python, firstline=35, lastline=109,caption={The code for the function (lines 35 to 109 of the program).}]{Automated_Error_Checks.py}
	\label{error_check_1}
	\newpage
	
	First, this function has two arguments:
	\begin{enumerate}
		\item var (the variable you introduce)
		\item condition (a list of True/False values that determines what type var should be)
	\end{enumerate}
	\vspace{3cm}
	
	Then, let's check the order of types in the function:
	
	\lstinputlisting[language=Python, firstline=45, lastline=46]{Automated_Error_Checks.py}
	
	The types follow this order:
	
	\begin{enumerate}
		\item int
		\item float
		\item complex
		\item tuple
		\item list
		\item dict
		\item (str) (opcional)
	\end{enumerate}
	
	There can be at most 7 boolean values that determine types by default in the program (although you can add more if you reprogram it).
	
	If there are more than 7 values (or the number of elements in condition), those will be omitted. You can only choose one type to check at a time without getting an error.
	\newpage
	
	After a lot of checks for the conditions variable, a general exec() check will be executed, which contains the next logic:

	\lstinputlisting[firstline=101,lastline=106]{Automated_Error_Checks.py}
	
	Here, although in a difficult way, the program checks for each element if the variables is or not its adequate type.
	
	If not, the program returns a TypeError.
	In this context, exec() is not a security issue, because the string which is passed through it is regulated and has been checked.
	
	Next, for more general use cases, ConditionCheck():
	\newpage
	
	\subsection{ConditionCheck()}
	\lstinputlisting[firstline=109,lastline=178]{Automated_Error_Checks.py}
	\newpage
	
	Instead of using two arguments, ConditionCheck() utilizes 3, which are:
	
	\begin{enumerate}
		\item var (the variable)
		\item conditions (a string with all the conditions var has to follow)
		\item severity\_mode (it indicates the program what it has to do when an error occurs)
	\end{enumerate}
	
	There are three main features the function presents:
	
	- The conditions must not syntax errors (they work exactly like other error handling).
	
	- You can fuse one or more conditions with \&\& (AND operator) or || (OR operator).
	
	- You can adjust the severity mode to 1 (returns False if condition is not met) or 2 (raises TypeError when condition is not met).
	
	The last function is this one:
	
	\newpage
	\subsection{BinaryToTruthList()}
	\lstinputlisting[firstline=11,lastline=35]{Automated_Error_Checks.py}
	
	It transforms a binary number (made with 1s and 0s) to a list with True or False values.
	\label{binary}
	
	\newpage
	\section{Complex functions: AutomatedBinaryToTruthList(), AutomatedErrorTypeFinder(), AutomatedConditionCheck()}
	\label{complex}
	
	These functions are automated versions of the previous basic ones, so to understand these one learning the simple ones is a must.
	
	\subsection{AutomatedBinaryToTruthList()}
	
	\lstinputlisting[firstline=179,lastline=198]{Automated_Error_Checks.py}
	
	In this function, the argument is a list with multiple binary numbers, which are converted into boolean value lists.
	
	\newpage
	\subsection{AutomatedErrorTypeFinder()}
	\lstinputlisting[firstline=200,lastline=230]{Automated_Error_Checks.py}
 	
 	The arguments in this function are lists, which are these ones:
 	\begin{enumerate}
 		\item variables (the list of variables)
 		\item condition\_batch (a list with multiple boolean value lists OR binary numbers)
 	\end{enumerate}
 	
 	For each variable, there must be a condition list. If that is not the case, it is a SyntaxError.
 	
 	Then, each variable will be checked and it will return True if no error occurs.
 	
 	\newpage
 	\subsection{AutomatedConditionCheck()}
 	\lstinputlisting[firstline=231,lastline=270]{Automated_Error_Checks.py}
 	
 	There are three arguments in this function:
 	\begin{enumerate}
 		\item variables (a list with all the variables)
 		\item condition\_batch (a list with all the conditions)
 	\end{enumerate}
 	
 	There must be a condition and severity mode per variable, if that is not the case, it is a SyntaxError.
 	
 	The results will be returned if there are no errors.
 	
 	\newpage
	\section{Use in Python3 code example.}
	
	\lstinputlisting{test.py}
	\vspace{3cm}
	\hspace{-0.5cm}ValueError: The conditions (var != 0) are not met with var = 0.
	
	It successfully predicts the edge case, without making another error take its place.
	
	You only need two lines of code for normal error handling.
	
	\label{example}
	
	\section{Credits}
	
	The author's Github page: \href{https://github.com/CyberCoral}{https://github.com/CyberCoral}
	
	\hspace{-0.5cm}The repository url:
	\href{https://github.com/CyberCoral/Error_Type_Checker}{https://github.com/CyberCoral/Error\_Type\_Checker}
	
\end{document}